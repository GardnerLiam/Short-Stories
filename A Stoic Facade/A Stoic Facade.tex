\documentclass[a4paper, 12pt]{book}
\usepackage[utf8]{inputenc}
\usepackage{authblk}
\usepackage{geometry}
\usepackage{setspace} 
\usepackage{amsmath}
\usepackage{textcomp}
\geometry{
 a4paper,
 total={170mm,257mm},
 left=20mm,
 top=20mm,
}
\title{A Stoic Facade}
\author{Liam Gardner}
\date{\today}
\doublespacing
\newcommand\tab[1][1cm]{\hspace*{#1}}
\begin{document}
\maketitle
\tab
Students walk onto the bus. They’re always so loud, especially during this time. School has ended and now they’re angsty and wanting to talk about all the new gossip. Three semi-familiar faces walk onto the bus. One of them is often silent and reminds the others to calm down and lower their voices. I like him. He keeps them in check. That being said, I also enjoy hearing the gossip. One day they’ll learn that all the stuff they complain about doesn’t really matter.
\newline
\tab
``He’s such a bitch.'' One of the girls says, ``I can’t believe he did that!''
\newline
\tab
``I know! And to think he hit on both of us as well even though he has a girlfriend!'' the other girl continues.
\newline
\tab
In the middle of them is a tall, thin, middle eastern looking boy with black, oily hair, listening as objectively as possible and trying to give advice without harming any party involved. He’s a stoic fellow. ``Calm down,'' he says, moving his hands slowly in a downwards gesture. ``I’ve heard that his relationship isn’t going that well. He might end it soon.'' His voice is smooth and originates from his diaphragm -- which is uncharacteristic for their generation -- resonating with his companions.
\newline
\tab
``Still! He has no right to hit on us like that!'' One of them says in an annoyingly high-pitched, nasally voice.
\newline
\tab
``She’s right! Especially since he hit on both of us! Unacceptable!''Their similarly-voiced friend responds with an equal amount of annoyance and disappointment.
\newline
\tab
``Are you sure he wasn’t trying to get some kind of emotional support? He doesn’t seem like the kind of guy to hit on two girls at once. Both of you know that.'' His reply seems in defence of the absent party, yet the two girls seem to accept his wisdom. There’s something different about him today. There’s something -- some pain in his eyes, the kind of pain caused from staying up all night alone with your thoughts. I know it all too well.
\newline
\tab
``You might be right; he may have just wanted to talk to someone.'' One of the girls responds, accepting but dissatisfied by his reasoning. ``You’re so in control of your emotions that I feel safe talking to you, ya know? Like you’re so -- ''
\newline
\tab
``Enough.'' The boy whispers clearly stressed about something.
\newline
\tab
``What did you say? Speak up Anshar.'' That name does sound mildly middle eastern, he might be Persian.
\newline
\tab
``I said enough!'' He screams at the girls. By this point everyone on the bus is staring at him. The bus driver is looking through one of his mirrors to see what’s happening. ``What the fuck do you know about \textit{my} emotions? Have you \textit{ever} let me open up to you about anything? What fucking right do you have to tell me about my emotions? Everyone always says that. \textit{I’m in control of my emotions. I’m so good with taking bad news.} Who the fuck told any of you that\textinterrobang You have no idea what it’s like, do you? To have your own behaviours and ideologies shoved down your throats? To be completely incapable of asking for help because \textit{it doesn’t fit you}, do you know what that’s like?
\newline
\tab
``Anshar, calm down.'' One of the girls says softly. ``You’re making a scene.''
\newline
\tab
The bus stops. The driver opens his door and steps out. ``Is everything okay?'' he asks in a surprisingly booming voice. The boy looks at the bus driver, met with disdain and walks off the bus.
\newline
\newline
\tab
``God dammit!'' I mutter to myself. ``They always say that. Always!'' I feel tears start to well up in my eyes. Quickly, I move my hair in front of my face to cover my now streaming tears. People look over at me with caution and fear, as if trying desperately not to approach me -- like I’m some kind of monster.
An older looking man approaches me. We make eye contact and I can see in him the will to console me. Another bus comes and I get on, pretending not to have noticed him. I tap my card to the bus’s scanner and sit in the back corner. Alone, just like I’ve always been.
\newline
\tab
I get home much later than normal. Without speaking a word to my parents, I make my way to my room and lie on my bed. Though I’m staring at my nightstand, my eyes don’t see anything. They’ve become glassed, void of any signs of life. I lie there, numb from the tears and forced into contemplation once again.
\newline
\tab
\textit{People always say that I’m good with my emotions, that I’m understanding and mature -- that I’m stoic. I hate that. They’re wrong. It’s their fault I’m forced to act like this. I hate living like this. I hate living.} I feel too tired to cry. Instead I lie there with movement mimicking that of a dead person. There’s nothing here for me anymore. I don’t need to stay awake.
\newline
\tab
I fall asleep, praying I don’t ever wake up. I’m not sure how long I sleep for. At the edge of consciousness, I feel myself pulled into an alternate reality. When I wake up, I’m seated in a church, on a mahogany bench at the back. I’m twelve. There’s a massive crack starting near my outer-left thigh that continues across the entirety of the object. Staring at it, the bench looks like it’s going to break. I cling to my father’s arm and he places a hand on my shoulder and embraces me. We’re at my grandmother’s funeral.
\newline
\tab
I have no control of my body. I’m crying, one of the few times it’s socially acceptable for me to cry. As the priest gives his eulogy, nobody listens. We’re all thinking about our time with my dead grandmother. She was the only person I could cry to. She understood that everybody has trouble controlling their emotions and would let me talk to her about anything. She was who I wanted to be for my friends. I lost my grandmother, and with her my ability to express myself. I became the person you can rely on during hard times and could come to for advice or to vent. What I didn’t realize was how much I had needed that.
\newline
\tab
I woke up thinking about my grandmother. She was the only person I could ever talk to. Her death still haunts me. This isn’t the first time that I’ve dreamt of that scene. I can always hear the priest’s voice, yet I’m never capable of understanding a single word. Everything’s in the background. Even now, nothing’s changed. As I lie on my bed, I’ve begun to realize how separated I am. My phone is off, no notifications to bring me back to the real world. My room is pitch black. Nobody disturbs me. My parents haven’t noticed something’s changed, and nobody’s going to inform them of what happened. I’m cut off from the world.
\newline
\tab
After getting off the bus, there are people who clearly heard me screaming. With tears streaming from my eyes, I move my hair in front of my face to hide them. People see me walking -- and probably see my crying as well -- and avoid me. This is different from reality. I’m watching this separated from my body. Everything’s in third person. That older looking man approaches me. We make eye contact. He sees the hurt in my eyes and comes over to help me. ``What’s wrong kid?'' he asks in a rough but well-projected voice.
\newline
\tab
``They don’t understand what it’s like for me. They vent and complain to me day after day and they always make me feel as if I can never vent to them or talk to them about my emotions.'' I say, holding back sniffles. It feels pathetic to watch, especially since I have no control of what my body is saying.
\newline
\tab
``You wanna talk about your emotions? What kind of guy are you? You \textit{shouldn’t} talk about your emotions like that. They’re for you, not others. Stop being a burden to them!'' He went from sympathetic to disgusted. At that moment, I realized I can’t see his face. He has no face. No eyes, nose, or mouth; just a blank slate. He’s mostly bald. Bits of light grey hair are scattered across his head in a dismal sort of look. Aside from that, his body seems much younger than his head. The suit he’s wearing doesn’t make him look older either. There’s just something about him that gives the impression of an old man, which only further aids his disappointment in me.
\newline
\tab
Even without a face, he wears his body with a disgust that begins to infect me just as the bus appears. I hop on, tears continuing to stream from my eyes. The faceless bus driver does nothing as I board without paying. As he closes the door to the bus, his body radiates hostility towards me. Panicked, I sit in the corner of the bus, waiting for it all to end.
\newline
\tab
I once again wake up. I hadn’t realized that I had fallen asleep. A feeling of disingenuity fills my body. What right do I have to be brooding over this bullshit? I’m useless. Even now, I spend my time lying here when I should have listened to them. I’m a burden like this. I’ve always kept to myself, what right do I have to overreact now? They must’ve been so embarrassed, ashamed to have been seen talking to me. They’ve probably already talked about it with their friend group. By now -- whenever that is -- everyone knows about it. My portrait of the stoic kid is probably shattered. I’m useless. I’ve become a burden they carry around with them. I’m a burden.
\newline
\tab
When I open my eyes, I’m standing on the edge of a cliff. Staring at my bare feet, I’ve noticed that it’s raining -- hard. The ground around me is wet, turned into the kind of mud I can feel myself slowly sinking into; my toenails are going to get dirty. The wind gets more and more intense as time passes. As I look up, I’m met with a distant hurricane. For a moment, I’m frozen in fear.
\newline
\tab
All around me, things become chaotic. As the hurricane approaches, the autumn-coloured trees dance in a fearful summoning ritual. It becomes harder and harder to tell where I broke off from the trail. Even though I’ve been here a million times, I don’t remember how to get back. I think about all that’s happened in my life. ``I’m only in high school, why should I die...'' my thoughts trail off as I stare at the impending doom heading my way. I don’t feel scared, more indecisive. I’ve imagined dying in over a thousand ways. Various disasters, to forms of suicide, to a stroke or a heart attack. This isn’t a new concept for me. I’ve always tried to escape my emotions, always being told that I keep to myself, that I'm very stoic. These offhand comments made me the way I am. I’ve always hated living like that. Even on the bus, I was never able to fully express myself. I doubt I’ll ever be able to.
\newline
\tab
Then again, they’re not bad people. For all the times I’ve helped them, they might be understanding and help me. I’m sure they saw it too though. I wanted to scream at them and say ``My stoicism is a lie! Stoicism is a lie!'' but I couldn’t muster the ability to say it. I don’t know what’ll happen to me. Now, I’m scared. As I watch the hurricane quickly approach the cliff, feet sunken into the ground, I’m not sure if it’s worth the struggle to run away. I mutter to myself. ``I don’t know what to do.''
\end{document}