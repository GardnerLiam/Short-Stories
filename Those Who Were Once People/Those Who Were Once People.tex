\RequirePackage{fix-cm}
\documentclass[a4paper, 12pt]{book}
\usepackage[utf8]{inputenc}
\usepackage{authblk}
\usepackage{setspace} 
\usepackage{amsmath}
\usepackage{textcomp}
\usepackage{amssymb}
\usepackage{geometry}
\usepackage{textcomp}

\usepackage{fontspec}

\geometry{
 a4paper,
 total={170mm,257mm},
 left=20mm,
 top=20mm,
}
\title{Those Who Were Once People}
\author{Liam Gardner}
\date{\today}
\doublespacing

\newfontface{\cuneiform}[Scale=MatchUppercase]{SantakkuM}
\DeclareTextFontCommand{\textcuneiform}{\cuneiform}

\newcommand\tab[1][1cm]{\hspace*{#1}}

\begin{document}
\newcommand{\AmaGi}{$\stackrel{\hbox{\fontsize{15}{60}\selectfont ama-gi}}{\hbox{\fontsize{30}{60}\selectfont {\symbol{"120BC}\symbol{"12104}}}}$}
\maketitle
\tab
“That’s all everyone’s been talking about for the last four months!” I say with more annoyance in my voice than initially intended. “Even if that kid said he performed the ritual, that doesn’t mean it’s true. He’s probably lying about it.”
\newline
\tab
“That kid was missing for an entire month. He was found with an arm and eye gone. You really think he’d have a reason to lie?” 
\newline
\tab
“The ritual itself is full of inconsistencies. Why would you inscribe a Sumerian word onto a stone, then pray to a Roman god?” I reply. These conversations are getting increasingly annoying. “What’s the point of it being an old building anyway? Or why does it have to be at midnight? Shouldn’t a time god be able to move freely throughout time?”
\newline
\tab
He answers with a smug annoyance in his voice. “Janus is the god of doorways and passages idiot. He’s not a time god.”
\newline
\tab
I sigh, realizing why plot-holes like this manage to continue to exist. “Janus”, I begin reading from the Wikipedia page of my phone, “is the god of beginnings, gates, transitions, time, duality, doorways, passages, and endings.”
\newline
\tab
“If you’re so confident that the ritual is fake, why don’t you do it? The school used to be an old church, right?”
\newline
\tab
I wonder how much trouble I’d be in if my parents figured out that I’m spending the night at school performing a weird ritual. “Yeah, yeah, fine. I’ll prove you wrong. You owe me \$50 if it doesn’t work.”
\newline
\tab
He looks at me in shock. “For what\textinterrobang” he almost shouts.
\newline
\tab
I turn to him before entering my next class. “For wasting my time.”
\newline
\tab
Annoyingly, he follows me in. “How am I supposed to get that money?” He asks in a begging voice that only serves to further infuriate me.
\newline
\tab
“If you’re so confident it works, you don’t have to worry about getting the money.” I retort, grinning. The teacher walks into the classroom causing him to leave, so as to not attract unwanted attention. 
\newline
\tab
Classes continue on as usual. People spend their conversations purely on the topic of that urban legend popularized on the internet. The day goes on and I do everything in my power to avoid these types of conversations, which typically leads to avoiding most people entirely. Unfortunately, that particular conversation between classes led to a decent chunk of my grade thinking I was bragging about me going to perform the ritual. I guess I don’t really have much of a choice about backing out now.
\newline
\tab
The day moves onward as more and more people find out about how I’m going to perform the ritual. I hear some people talk about someone else also going to do it as well, but I don’t really care enough to listen in. The day ends and I go home to pack my things. On the bus, a younger student hands me a stone with letters carved into it. “For the ritual,” he says quietly and embarrassed. “Good luck.” I nod, staring at the cuneiform inscription, \textcuneiform{\AmaGi} the Sumerian word for freedom. I almost begin to laugh. The more I think about it, the more it seems like something an edgy teenager would try to run away from their life and responsibilities.
\newline
\tab
I get home, enter the kitchen, and grab a pack of candles and a lighter. Heading up to my room, I grab a few blankets to sleep on, choosing to ignore taking a pillow for the sake of remaining relatively light. Chances are, it’ll be easier to stay at school for the night rather than to bus all the way back home. It’ll probably also be safer as well. Heading towards the bus stop, the thought slowly creeps up in the back of my mind that the ritual might actually work. I take a seat on an almost vacant bus and pull up the blog post on my phone. 
\newline
\tab
\textit{
Place the candles in a triangle, with the stone in the middle and say the words “O god of passage, strip us of our sins and bring us to the free world!” 
}
\newline
\tab
Reading over it, it also says there’s no way back. A wave of relief washes over me. There’s no way to know if it’s impossible to come back or not. You either don’t know, because you haven’t tried the ritual yourself, or you know you can come back, because you tried it and returned to make the blog post. The more inconsistencies I find in this, the more reassured I feel about my success.
\newline
\tab
Walking into the main hall of the school, I find one other student who probably shares my intention. “Andraste!” I call out to her, walking over to see what she’s doing.
\newline
\tab
“Oh,” she replies, “I didn’t think you’d show up.”
\newline
\tab
“I didn’t really have much of a choice. This is probably the most effective way to get rid of the legend, right? Or at least in our school.”
\newline
\tab
“Yeah, probably. It’s only 21:00, what do you want to do for three hours?” She asks. If we share a common goal, we may as well work together, not that it’d help if I stated that. My stomach slowly starts to silently tell me that I haven’t eaten in a while.
\newline
\tab
“Have you had dinner yet?” I ask, getting increasingly hungry as we speak.
\newline
\tab
“Imagine that, the last thing that happens to me in this world is being asked out on a date!” she replies almost laughing. “No, I haven’t eaten yet.” 
\newline
\tab
I choose to ignore that comment, not wanting to start an argument. “Let’s get burgers then,” I say, “I haven’t eaten since lunch.” She nods and we head out to get dinner. Aside from a few staff, the burger joint is completely empty. It feels kind of eerie. There should be at least a few more people here at this time, right? We order our food and sit down. “What do you think of this?” I ask, attempting to distract myself from the unnerving atmosphere.
\newline
\tab
“Zius, you don’t really believe in all this, do you?” she asks, seeming almost concerned.
\newline
\tab
“My name is Ziustra, not Zius.” I reply with an almost instinctual hostility. After a couple seconds, I continue in a calmer manner. “No, I don’t believe in any of this. I just hate that it hasn’t died off yet.”
\newline
\tab
She smiles and nods at my response. “Good, we’re in the same boat then.” I could’ve probably told her that before we got food…
\newline
\tab
We walk back to the school, talking about all the inconsistencies we found in the ritual. Mixing of religions, not being able to get back, eventually, we found out that the post was copied from another, less-popular forum. Slowly but surely, time passes. Both of us get more and more anxious as we get closer to midnight. Neither of us bring up the possibility of the ritual being true, especially with the thought of being unable to return. The clock shifts to 23:55. “I guess we should get started,” I say with a mild tinge of regret in my voice.
\newline
\tab
I pull out three candles from my bag and place them in a triangle about 30 centimetres apart from each other. I reach further into my bag and pull out the inscribed stone given to me by that kid on the bus. Placing the stone in the middle of the triangle, inscription-side down, I look back at the clock to see how much time we have. 23:57. I take out the lighter – which in hindsight shouldn’t have been placed around blankets and candles – and light the candles. We wait for the clock to reach midnight before beginning our prayer. 00:00. “O god of passage, strip us of our sins and bring us into the free world.” One second has passed. Two seconds have passed. Three, five, ten, thirty; nothing happens. We look over at each other, holding our breath. I practically expect some sort of jump-scare, but nothing comes. We breathe a sigh of relief. We end up spending the night at school in makeshift sleeping bags made from the blankets I brought. I leave the candles as they are just to provide some light since the school’s lights are off.
\newline
\tab
When I wake up, Andraste’s sitting around the now-extinguished candles. I get up, slowly, and she snaps her head in my direction, looking at me like I’m a ghost. I begin to think that maybe I did or said something in my sleep.
\newline
\tab
There’s no sound. No hums from the air conditioner or lights. No insects or wind, nothing. I can hear the sounds of my breathing and heartbeat so clearly it feels like I have control of their pacing. I feel like I have to consciously think to keep my heart moving. “Do you… hear anything?” I ask slowly. Even the sound of my own voice feels off.
\newline
\tab
It takes her a moment to reply “Nothing.” I nod slowly.
\newline
\tab
We pack up the stuff into our bags and begin to walk towards the exit. “There’s something… off,” I begin speaking without really thinking. “not the silence but –”
\newline
\tab
“The atmosphere.” She stated in a grave and serious tone before turning back to me. Her next words were the only thing I didn’t want to hear. “It worked.”
\newline
\tab
I laugh to myself, “looks like I’m not getting that \$50,” she gives me a disgusted look but says nothing.
\newline
\tab
We performed the ritual in the corner of the school, on the second floor. It’s typically completely empty during the day, so just in case anyone else shared our thoughts, we wouldn’t see them. I’m sure there are even a couple students that didn’t know this place existed. In one of the main stairways lies confirmation of the ritual. A fully-clothed metal statue of a student looks towards the top-left of the stairway – away from us – with outstretched arms. “I don’t think… that statue’s… manmade.”
\newline
\tab
Andraste’s words only serve to terrify me further. She’s probably right, but I don’t want to believe it. I say nothing. Looking at the statue, the metal itself seems so polished and clean that it could be a mirror if it weren’t for the shape.
\newline
\tab
We move closer towards the statue, half-expecting it to move. It doesn’t. “Look!” I say, pointing at it’s feet. “A notebook.” We haphazardly rush down the stairs. Looking back at the statue in fear, I get a clear look at its – her – face. The statue’s face looks like that of a crying girl. Her teardrops solidified onto her face, her eyes are slightly slanted downwards, and her face looks… worn? No, that’s not right. I can’t really describe it, but it only serves to make things even more terrifying. Andraste’s right, that thing doesn’t look manmade. Nobody could sculpt a face that… powerful.
\newline
\tab
She picks up the notebook and begins flipping through it. I move to stand behind her just as she begins reading the last entry of what we’re now figuring out is a diary.
\newline
\tab
\textit{
It started a month ago. Everyone thought it was just a flu, another strand of the influenza virus. Our first hint should’ve been the vaccines – they didn’t work. A lot more people died, but that’s nothing in comparison to what’s happening now. To be honest, I’d rather be dead. Most people recovered on their own, it was really just the very young children and elderly who were at fatal risk. A week after people recovered, they would start turning into these metal statues. Nobody’s sure why, or even how, but we’ve discovered a few things:
}
\begin{enumerate}
\item \textit{The statues somehow block all technological communication. No radio, no wi-fi, nothing. Everything becomes distorted.}
\item \textit{The statues are virtually indestructible.}
\item \textit{The statues are not contagious and cannot spread the virus. I still wouldn’t go around touching them.}
\end{enumerate}
\tab
\textit{This is day 7 for me. On the off-hand chance that someone finds this, good luck.}
\newline
\tab
\begin{center}
\textit{April 11th, 2018.}
\end{center}
\tab
She drops the journal. “That was two years ago.” She says, looking at me.
\newline
\tab
I nod, picking up the journal and putting it in my bag. “Let’s look around the school more.” I say, trying to distract both her and me. She only nods, turning away from what once was a person and walking into the main hall of the school. There are more statues of students and teachers in the hallway and in the classroom. Every statue either looks like it’s crying or about to cry. Part of me hopes that the journal was a prank or joke, but I know nobody has the ability to create something this… real. We walk into the main office; the receptionist is a crying statue. I walk over to the front desk and pick up a calendar. “It looks like this all happened two years ago. At least that rules out time travel.” I sigh. This means that time is synced with our world.
\newline
\tab
Andraste looks at up at me. “If we didn’t time travel, that means the virus has had time to die off, at least in this area. We aren’t immune, but chances are we’re relatively safe from it.” I nod in agreement. Her stomach growls, the silence only serving to make the noise seem louder. She blushes in embarrassment.
\newline
\tab
“We need a food supply. If it’s replenishable, that’s probably better. We don’t know how…” I trail off, not wanting to finish that thought.
\newline
\tab
“You’re right,” she says, “let’s check the school garden.”
\newline
\tab
“They grow a few edible things, but we’d still need a way to get meat, right?” The garden wasn’t ever meant to be used as a food source in the first place. It was probably just meant as a way to judge the students’ level of maturity or responsibility. Either way, small amounts of food is better than nothing, so I keep my mouth shut.
\newline
\tab
“It’s still worth checking out,” I nod, and we head over to the garden. The moment we walk into the courtyard, I pause. \textit{There’s something wrong.} She walks over to the plants and lifts up a leaf. “These seem fresh. We should be able to eat these.” She says, turning to me with what almost seems like excitement in her eyes.
\newline
\tab
“Hey,” I say slowly. She looks at me quizzically. “Who’s been tending to the garden?” she lets go of the leaf, slowly walking towards me. “Let’s get out of here,” she says. She has that look in her eyes that makes me think she’s noticed something. I follow her inside, closing the door behind us. “Did you see something?” I ask, wondering why she’s so freaked out.
\newline
\tab
“Don’t panic,” she starts. I can feel my face slowly becoming pale. “You’re right. If it’s been two years since everyone’s turned to metal, those plants would’ve been overgrown. But they weren’t. The grass was cut, the plants were trimmed, everything was clean. Just like –”
\newline
\tab
“The school,” I say, catching on to her train of thought. “There’s someone else here.” She nods. We don’t move for a few minutes, individually trying to come up with our own ideas of what to do next. “We should stay in the school. We’ll have an easier time monitoring if anyone comes in or out. Also –”
\newline
\tab
She continues my thoughts. “We’re a lot less likely to see someone before they see us if we’re outside. We’d be like sitting ducks.”
\newline
\tab
“For now though, we still need food. Let’s go back out into the courtyard. Try and pick out the ones near the back, and only get one or two from each plant.” With that, we go back outside, trying to make as little noise as possible, while picking out as many vegetables as we can within the span of a few minutes before rushing back inside and placing down our goods.
\newline
\tab
Andraste sighs, “We’re going to have to cook most of these,” she says. Looking at our new food source, we’ve collected a few carrots, onions, potatoes, radishes, and leeks. I sigh as well.
\newline
\tab
“We should try the cafeteria,” I say. She nods in agreement. “We uprooted most of these. If whoever’s here goes to the garden, they’ll notice that we’re here. We should avoid the garden for the time being.”
\newline
\tab
“If the cafeteria’s anything like our school, the fridge will be empty. At least we’ll be able to cook these,” she says. We walk towards the cafeteria, carefully taking note of the crying metal statues as well as looking out for any movement.
\newline
\tab
We get to the cafeteria, open the door to the other side of the counter and move towards the deep fryer. She flips the power switch, causing the machine to start making a loud air-conditioner-like hum. “Shit! Shit! Shit! Shit! Shit!” she shouts, hurrying to turn off the power. “What the hell was that\textinterrobang”
\newline
\tab
“We’re so used to the background noise that being without it makes everything else seem louder. You noticed it when we first woke up too right? The sound of your heart beating so clearly it seems like you have to manually control it.”
\newline
\tab
“So what do we do now? Eat everything raw?” Andraste shouts at me.
\newline
\tab
“We have no idea who the person taking care of this place is or what their intentions are. They might actually want to help rather than kill us.” She seems doubtful of my words, but the ideas at least calmed her down. “Turn it on, I’ll wait by the entrance while you cook. If anyone starts coming towards us, I’ll let you know.”
\newline
\tab
“And then?” She replies annoyed. “What’ll we do if someone comes for us? What if they’re \textit{not} friendly? How can we even \textit{think} of fighting someone if we don’t know what weapons they have? Someone comes, following the sounds, we’re preparing to stab them with kitchen knives while they have a fucking machine gun!”
\newline
\tab
“Where would they even get a machine gun from?” I reply in a calm voice, which only serves to anger her more. “This is the safest option we have aside from eating an onion like an apple. Let’s go prepare the food first, then we’ll turn on the deep fryer when we’re ready. I’ll wait by the door and tell you if I see anything. If I see any movement, we’ll leave the deep fryer on and hide in a closet. If they have guns, at least we’ll stand more of a chance that way.” She doesn’t fight this argument, though I’m guessing more out of not wanting to fight than out of agreement, and we begin preparing our food.
\newline
\tab
“We could boil these and make a soup,” she says after a few minutes of silent chopping and peeling. “It’d probably be healthier too.”
\newline
\tab
I smile in agreement, relieved that I don’t have to watch out for a potential murderer just yet. “Finish up here, I’ll get the stove and water ready.” She turns back to chopping the last carrot as I fill a pot with two litres of water and place it on the stove. She comes over, pouring the copped vegetables into the water as I turn on the element. “Did we ever check the fridge?” I ask after half-covering the pot with a lid.
\newline
\tab
“I don’t think so,” Andraste replies, “Why? There’s nothing there normally. If this is a mirror of our world, there’d be nothing in it now, right?”
\newline
\tab
“If there’s someone here, and everything has power, why wouldn’t they use the fridge to store food?” Her eyes widen with realization, but not from what I told her.
\newline
\tab
She looks at me in that same way she did back at the garden. “If there’s someone here, and they use the cafeteria as food storage, this is probably the place we’re most likely to find them, right?” The water starts whistling, signifying the end of our wait. Andraste moves to turn off the stove. We carefully ladle our soup into two large bowls, grab spoons, and move behind a stairway at the other end of the hall. “We may as well leave everything there, if they look at the garden, they’ll notice that we’re here anyway.”
\newline
\tab
“Everything we do leads to us getting closer and closer to being noticed, right? If we’re going to stay here, we should stay in a classroom. That way we can at least lock the door.”
\newline
\tab
She ponders for a moment as to what our next move should be. “Let’s sleep in the science wing, it’s on the opposite side of the school, away from the garden and cafeteria.”
\newline
\tab
“If someone finds out we’re here, they’d search the entire school.” I reply, unconvinced that we should avoid our previous locations. “We should stay on the first floor, somewhere with a window we can escape out of and into the city if we’re found out.”
\newline
\tab
“The science wing should at least have supplies to make some sort of weaponry. We’d be more prepared for defence than we are now. Let’s at least stop by.” I nod, mostly out of not wanting another argument. As we finish the soup, I begin to wonder what materials she’d want from the science wing. We make our way up the stairs and towards the other side of the school, passing by at least ten other metal statues. Entering the chemistry lab and walking into the room labelled \textit{No Students Allowed}, I turn to her and ask, “what is it that you’re looking for?”
\newline
\tab
She opens a drawer filled with opaque containers. “Iron oxide, aluminium powder, and magnesium.” 
\newline
\tab
“For what exactly?” I ask slowly.
\newline
\tab
“Didn’t you ever pay attention in science class? I’m making thermite!”
\newline
\tab
I back out of the \textit{No Students Allowed} room slowly. “What are you going to use thermite for? I can’t see that being easily weaponized.”
\newline
\tab
“You know that thing some people do where they’ll draw a line of salt to prevent demons from walking in? Here’s the anti-human version,” she replies, smirking.
\newline
\tab
“Did you forget the tiny fact that we’re also human? That’ll kill us too!” I almost shout. 
\newline
\tab
She turns to me, walking out and holding three containers. “Hear me out,” she starts. “You were right. Sleeping up here makes us cornered. Having a window as an escape option is nice, but if we’re followed it becomes a test of stamina. If they have a gun, we automatically lose. Let’s assume they find us. The noise from trying to break down the door will wake us up and give us enough time to run? Then what? How do we make sure we aren’t followed? Between the window and door, we’ll draw a line of thermite that we can ignite of we hear someone. It’ll be bright enough to stop them from getting a good look at us as we run away.” It seems a bit overkill, but I don’t argue against the idea.
\newline
\tab
“Sprinkler systems activate from heat and smoke detection, so thermite should be able to trigger that as well.” I reply. “I never took chemistry though, so don’t expect me to help.”
\newline
\tab
She grins, as if hoping that I wouldn’t be able to help. “Don’t worry, just get the room ready. I’m assuming you know which one we’ll be sleeping in.” She says, focusing on sorting out how much of each material she’ll need.
\newline
\tab
“I’d rather not be alone with all the metallic statues,” I admit, slightly embarrassed. “There’s just something about the fact that they were all people. It’s kind of unnerving.”
\newline
\tab
She sighs, turning towards me and smiling. “It’s alright. Sit over there.” She says, pointing to the other end of the room. I smile back and make my way over as she begins to work her magic – or I guess it’s more science. It takes her roughly 15 minutes to finish making a bucket-full of thermite. 
\newline
\tab
I’m staring out the window as the sun begins to set. “We should probably stay in 107,” I say, “It has a decently big window on the opposite corner as the door. The room itself is also at a size where the distance between the door and window is fairly large, but the room itself isn’t big enough for us to use all of that thermite.
\newline
\tab
She nods and we head down the stairway to room 107. We open the door and find two statues at the from to the room staring at a wall with semi-outstretched arms. Andraste turns to me. “Are you going to be okay here?” 
\newline
\tab
Her question catches me off guard. After a second, I respond. “I’ll be fine. At least they’re not facing us.”
\newline
\tab
“Get the blankets ready. I’ll make the barrier.” I chuckle at the word barrier being used to describe a thin line of thermite. I set up two blankets near the centre of the window. She turns back to me. “You still have your lighter, right?”
\newline
\tab
I smile. “Shouldn’t you have checked before making the thermite?” I ask, only to get a glare telling me to answer the question. “Yeah, I still have it.” I reply, pulling out of my bag to show her.
\newline
\tab
We lock the door to the room and get in our poorly made sleeping bags, laying there silent for 30 minutes. She’s not so close that I can hear her breathing or heartbeat, but she’s close enough that I can tell she’s still awake without having to turn my head. “Are you okay?” I ask softly.
\newline
\tab
“How could anyone be okay in this situation?” She replies. In a softer, almost inaudible voice, she continues. “I’m terrified.”
\newline
\tab
I almost reach out my hand but hold back the urge. “So am I.” I reply. “I never thought that the ritual would actually work.”
\newline
\tab
“Yeah. I just wanted everyone to stop talking about it.” She pauses. “I wonder what they’re all doing right now?” Her tone terrifies me. It makes her sound like she’s on the verge of dying. I can’t stop myself. I reach out and hold her hand – more for my sake than hers. We stay like this in silence for as long as I retain consciousness.
\end{document}