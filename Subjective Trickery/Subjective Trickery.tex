\documentclass[a4paper, 12pt]{book}
\usepackage[utf8]{inputenc}
\usepackage{authblk}
\usepackage{setspace} 
\usepackage{CJKutf8}
\usepackage{geometry}

\usepackage[pdftex,
            pdfauthor={Liam Gardner},
            pdftitle={Subjective Trickery},
            pdfsubject={N/A},
            pdfkeywords={N/A},
            pdfproducer={N/a},
            pdfcreator={pdflatex}]{hyperref}

\geometry{
 a4paper,
 total={170mm,257mm},
 left=20mm,
 top=20mm,
}
\title{Subjective Trickery}
\author{Liam Gardner}
\date{\today}
\doublespacing
\begin{document}
\newcommand\tab[1][1cm]{\hspace*{#1}}
\maketitle
\tab
Math classes are always so boring. The teacher’s voice is always monotone and the stuff he teaches always takes forever to get through. The idiots in the class, which make up most of the students here, understand less than a dead fish if it were asked to explain how to fly. Maybe I’m being a tiny bit overdramatic, but it’s in my nature.
\newline
\tab
I open my wallet and pull out a coin. I flick the coin into the air over and over again. The teacher finally notices and pauses his almost-lesson. ``Would you mind not doing that in my class?'' He sounds annoyed. I guess that ping is finally starting to get to him. I smirk, this should be fun.
\newline
\tab
``Why don’t you guess? If you get it right, I’ll put the coin away.'' The class looks back at me, at the back-middle of the classroom, typically where the quiet kid would be, with their bodies’ half turned all the same way. It’s almost movie-like. Some people snicker and whisper to one another, others are silent and wide-eyed. The teacher has lost control of his class. I hold the coin in my hand and gaze back at my audience. There’s this little trick I can do that only a few people know. I make eye contact with one of my classmates. He’s smiling, and thinks the coin is rigged; a dual-headed coin. The couple people around him, presumably his friends from all their whispering, also think the coin is rigged. The blond, short girl with glasses thinks the coin is two tailed instead of two headed. The tall, lanky guy behind her thinks the coin is weighted more on one side so you can flip in in a specific way to always get the same result. Creative, but I don’t think anyone else is thinking of that -- or that it would even work in the first place. I scan the rest of the room, there are five other people who think the coin is rigged. Perfect.
\newline
\tab
``Heads. Flip.'' The teacher says, knowing that he’ll get nowhere arguing with me. I flip the coin in the air and focus on three people’s subjective understanding of the object.
\newline
\tab
\textit{Collapse into reality.}
\newline
\tab
The once fair coin is now dual sided with tails. I pluck the coin from the air and smack it onto the back my hand. I remove my palm from over it. ``Tails.'' I say, smiling and staring into his eyes with superiority. ``Teach.'' The people around me confirm that it was indeed tails.
\newline
\tab
He groans. ``Let me see that!'' He semi-shouts, walking over to me. I slide through the other people’s understanding and find a few who think it’s a fair coin. Just before he takes it from my hand, subjectivity becomes reality. The coin is fair. He inspects it with skepticism. I can see his reality, even in his understanding it’s a fair coin. He sighs, handing it back to me before walking over to the front of the class and continuing to teach.
\newline
\tab
A few people glance back at me in amazement. They probably weren’t expecting me to win. Even if I hadn’t cheated, it still would’ve been a fifty-percent chance. Mind you, I’m in a math class with people who don’t understand math, so fifty-percent probably goes over their heads. I continue to flick the coin into the air. The ping annoys the teacher more and more. Eventually, he breaks a piece of chalk on the blackboard. I seem to have mentally broken him. ``Where’d that two come from?'' I ask, trying to be discrete in pointing out his mistake, ``In step seven.''
\newline
\tab
He looks at me, then back at the board, then back at me again. ``Look at the previous steps, we divided out the four. If you were paying attention instead of flipping that damn coin, maybe you’d understand better.'' He says with superiority and confidence.
\newline
\tab
``I don’t have to pay attention to know that four over three isn’t two.'' I respond, amused at his arrogance.
\newline
\tab
``What are you...'' He turns back to the board and mumbles something to himself before erasing everything. Some students whine in protest, but nobody actually stops him. ``That question is for homework.'' He says promptly before returning to his desk and placing his head on folded arms.
\newline
\tab
He’s been like that for fifteen minutes now so he’s crying or sleeping... or both. Either one wouldn’t surprise me. I quietly pack my bag and leave the class. He doesn’t notice. A couple of students see me leave but say nothing. I walk down the hall, preparing to leave school two periods early. Nobody says a word, the halls are silent. Every class is either really good at listening or writing some sort of test.
\newline
\tab
I leave and take a shorter route home. Normally, I don’t go this way because... dammit! Five people come out of nowhere, grinning. Three of them are holding baseball bats. It’s best to typically avoid these people, for obvious reasons, but I assumed they’d be at school. Now that I think of it, they don’t seem like the type to go to school in the first place. I open my bag and slowly pull out an airsoft gun. I’ve scraped off the orange tip, so it looks like a real pistol. I keep it hidden from them and feign defenseless. ``You’re gonna do two things for us.'' One of them starts, grinning menacingly. ``Drop your wallet and run.''
\newline
\tab
I smile and reveal my fake, unloaded weapon. ``See, I ain’t gonna do shit for you.'' I point the gun at one of them. Two people take a few steps backwards. I found my targets. I peer into their worlds. They both assume the gun to be real and fully loaded. They start backing off. Subjectivity turns to objectivity and the item in my hand is no longer an unloaded airsoft gun. ``Drop your wallets and run.'' I say with the same grin they wear to scare me. A bit of extra cash never hurt anyone, right?
\newline
\tab
One of them smirks. ``Ya think we’re scared of an airsoft gun? Did you forget to scrape off some of the orange?'' He says, slowly beginning to approach me. The other two calmed down and seem more confident now. The superposition of subjectivities have already collapsed into reality. This is no longer an airsoft gun. I point the gun at one of their arms and shoot. My ears start ringing. That was the first time I’ve ever fired a real gun.
\newline
\tab
They scream in pain and drop the bat. Blood seeps onto their clothing and they look at me in shock. ``Aww, did you think you were smart? Whoop de fucken’ doo, it’s not an airsoft gun.'' I say, smirking. They all drop their wallets and run, as instructed. I pick them up, take the money out of them and throw them in the garbage. Walking home, I dispose of the ammo and gun in two separate garbage cans, about five blocks apart from each other.
\newline
\tab
I walk into an empty house, drag myself over to my bed and lie down. It’s been a long day, even though it’s only 14:00. There’s an untouched apple on my nightstand. I turn my head and glance at it contemplatively. Right now, some people would say there are two versions of that apple. There’s the actual apple and there’s my subjective understanding of it. I think these people are called new realists. They combine the ideologies of metaphysics, where only the apple exists, and constructivism, where only my interpretation of the apple exists. It’s funny how so many people have written so many existential theories and most of them are wrong. I’m living proof of that. Technically, I don’t know which one is true, but I’ve narrowed it down to either new realism or constructivism.
\newline
\tab
When I peer into other people’s realities, the object I’m affecting always exists in their subjectivity and mine. So, either constructivism is true and I’m affecting everyone’s interpreted realities, or new realism is true, in which their subjectivities are applied to the metaphysical object which in turn recalibrates their subjectivity. They both seem just as believable, but I’ve conformed to new realism. Though I can’t prove this, I’d like to believe that objects can exist without interpretation. I could make that apple fresh just by convincing people I took it out of a fridge two minutes ago. Their understandings of the apple would be that it’s not rotten inside, and I could turn that into reality. Not many people know about my unique ability. The two I’ve told pretty much lose their decisiveness around me. They think that because of me, a simple coin can exist in a multitude of ways at once and are thus never sure about what the coin actually is. We slowly grew apart and I don’t talk to them much anymore. I think one of them even moved to a different country.
\newline
\tab
Reality is undecided with me around. There are scenarios I’ve wondered about, but I’ve never been able to verify any of them. Let’s say I’m walking at midnight with some friends and we found a run-down building with a couple homeless people living in it. If I were to tell them it’s abandoned, and they believed me, would making that reality delete the homeless people living there? Would they be moved and if so, then to where? Would they die, or cease to exist? If they cease to exist, would they have ever existed in the first place? Playing with lives like that is too hard for me mentally. If I were to kill someone, even accidentally, it would probably consume me. I’ve restricted myself to small objects for that exact reason, but even then, I’m still able to cause serious damage. I turn my head to the ceiling and stare up, slowly voiding myself of thought and falling to sleep. 
\newline
\tab
Going to school always takes a tremendous amount of effort. On the one hand, I really don’t have to go unless I have a test or have to hand in an assignment. On the other hand, there’s not much to do here either. I’ll have to order a new airsoft gun online soon, though that’s not going to take an entire day. It’s probably worth going to school. Typically, I’d say the pros outweigh the cons, but there aren’t really pros or cons in either situation.
\newline
\tab
I head to school, apathetic as usual. Typically, my parents would force me to go to school, but they decided last week they were going to go on vacation without me. They probably told me about it a while ago and I forgot or wasn’t paying attention. I walk to my English class and glance at the whiteboard. In big letters on the centre, the word ``Substitute'' is written. I look at the teacher’s desk and smirk; it’s my math teacher. This is going to be a fun class. He sees my face and turns pale for a few seconds before drowning in infuriation and hostility.
\newline
\tab
Most of the students currently here stare at our interaction with confusion. There are two or three from my math class, who’ve gotten used to me mentally torturing this particular teacher. ``For some reason, I feel as if you don’t like me. Have I done something to offend you, \textit{dear substitute}?'' I mildly fake a British accent for dramatic purposes, which only serves to infuriate him more.
\newline
\tab
``Get out! Go to the principal’s office and sit there until I come and explain everything. Then we’ll see if you still have that hubris.'' He spits onto the floor as he screams. It’s almost unnoticeable if you don’t know to look for it. Part of me hopes he slips on the ground and falls, though that’s probably too harsh a wish.
\newline
\tab
I look at the clock. ``Class starts in two minutes. If I’m at the principal’s office then I’m not here, yes?'' He nods slowly. ``If I’m not here, I’m absent. If I’m absent, you have no power over me. If you have no power over me, I won’t be going to the principal’s office.'' I explain my situation to him and give him a few seconds to digest the information. ``So, pick your poison.'' I say after a minute passes in complete silence. Not even whispers from other students could be heard.
\newline
\tab
``You’re absent. I can’t wait to see what your parents think of this.'' He says with a smirk on his face.
\newline
\tab
``My parents are on vacation. I’m alone.'' I reply, smirking back. ``When I said you have no power, I wasn’t kidding.
\newline
\tab
He turns to his desk and puts his and hand on his chin, stroking contemplatively. ``Sit down.'' He finally says after what feels like a month of me standing there. I swear I saw seasons change from the window as he was deciding. ``I will take your wallet forcefully if you pull out a coin.'' I smile at him like an innocent child. The bell rings as the last few students enter the class. He grabs a sheet of paper and clears his throat. ``Okay class, you’re meant to be working on some sort of project.'' Oh. I forgot about that. It’s a good thing I came to school today. ``I’ll be collecting parts of it as the class progresses. I’ll write down the times your teacher gave me for you to finish certain sections.''
\newline
\tab
He reaches for a marker. ``That one’s dry.'' Just as I started saying that last word, most of the students have already completed my sentence in their heads. It takes a bit longer for certain students, who waited brainlessly for me to finish my sentence before coming to their own subjective understanding. I reap every students’ individual subjectivity and force the marker to lose its moisture.
\newline
\tab
``I’ll throw it out later.'' He says apathetically, picking up a different marker and beginning to write out the project’s due dates -- or I guess it’d be due times -- on the board.
\newline
\tab
That’s probably the only fun I’ll have this class. It’s easy to mess with him like that for almost every action he takes but attacking him constantly ruins the fun. Instead, I do the tasks we have to hand in today. He collects everything from me on time, getting more and more surprised every time he sees my completed work. When the bell rings, he holds me back. He stares at me with a sincere look in his eyes. ``Why can’t you be like that in my class? You have potential, I can see it! You also certainly aren’t the worst student in that class. So, why don’t you work like that when I teach you?''
\newline
\tab
``One simple reason. I don’t feel like it.'' In a world where I can make anyone’s shared subjectivity into reality -- even misunderstandings -- almost all of my actions are driven by self-interest. Everything boils down to whether or not I want to do something. This power creates a disconnect between me and other people. My inability to properly explain this skill leads to people fundamentally being incapable of understanding me. He happens to be one of those incapable people. I chose to work today because I thought it was in my best interest, his class isn’t like that. Out of everyone I know, I’m the only person who can do that, so nobody else can understand what it’s like.
\end{document}